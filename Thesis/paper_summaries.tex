\documentclass{article}
%\usepackage{syntonly}
\usepackage{fancyhdr}
\usepackage{amsmath}
%\syntaxonly
\usepackage{titling}
%\usepackage{apalike}
\usepackage[authoryear,round]{natbib}
\newcommand{\subtitle}[1]{%
  \posttitle{%
    \par\end{center}
    \begin{center}\Large#1\end{center}
    \vskip0.5em}%
  }
\setlength{\parindent}{0pt}
\setlength{\parskip}{1ex plus 0.5ex minus 0.2ex}
\fancyhf{}
\renewcommand{\headrulewidth}{0pt}
\fancyfoot[LE,RO]{\thepage}

\begin{document}
\pagenumbering{roman}


\title{Eigenvector analysis for the ranking of control loop importance}
\subtitle{Research summaries}
\date{}
\author{Simon Jacobus Streicher}
\maketitle
\thispagestyle{empty}

%\newpage
%
%\tableofcontents
%\thispagestyle{empty}

%\frontmatter

%\mainmatter


% Research groups identified:

% Nina Thornhill
% Mano Ram Maurya

\newpage
\pagenumbering{arabic}
\pagestyle{fancy}

\section{Prioritizing control loops for maintanance}

%~\citep{Rahman2011}

\section{Finding signed digraphs}

The connectivity between process variables can be determined in at least two different ways:
1) Determining causality using data-driven methods and
2) defining connections by making use of known process topology (referring to physical and logical connectivity) information contained in P\&IDs (pipeline \& instrumentation diagrams) which are normally kept well organized and reasonably up to date.

\subsection{Data-driven methods}
%\cite{Bauer2005b}
%~\cite{Bauer2007}
%~\cite{Bauer2008}

\subsection{Knowledge based methods}


\subsection{Hybrid methods}
\citet{Yim2006}:

Software that can be used to make object-oriented representations of processes:

CosmosPT from Innotec
Intools or SmarPlant P\&ID from Intergraph



\citet{Thambirajah2009}:

\citet{Thambirajah2009} builds on the work of \citet{Yim2006}.

Copied from abstract:
Both quantitative and causal maps and digraphs can give spurious results.
The amount of spurious results can be reduced by combining basic and readily available information about the connectivity of the process with the results from causal measurement-based analyses.

Specifications and standards for engineering data exchange include DIN V 44366 IEC/PAS 62424 which specifies the CAEX (Computer Aided Engineering Exchange) schema as well as the ISO 10303-221 and ISO 15926-7 standard which is represented by XMpLant.

Forward and backwards tests on networks derived from known topological information used in the testing of hypotheses derived from data based methods.

Transfer entropy is used to analyse fault propagation in ABB's Loop Performance Monitor tools.

ABB's Plant Disturbance Analyser (PDA)



\section{Signed digraphs - theory and application}


\citet{Mauraya2003a}
\citet{Mauraya2003b}:
\citet{Mauraya2003b} gives a number of algorithms for deriving digraphs as well as signed directed graphs from differential and algebraic equations.
Inferring the effect of deviations in nodes on the rest of the network by qualitative algebra is also discussed.
One of these methods focus on the initial response only.
Finding the length of paths (degree of separation) between nodes by looking at the matrix exponent of the adjacency matrix is also discussed.
See also \citet{Jiang2009}

A number of means of reducing the amount of spurious results are discussed.
One such method is the generation of noncausal redundant equations.
Generating redundant equations by algebraic manipulation on some of the original system equations helps to reduce the number of spurious solutions if it contains a subset of variables different than he subset of variables present in the original equations from which the equation was derived.
Another means of reducing spurious results is to attach weights to the edges (archs) in the signed digraph so the relative order of magnitude of deviations in different system variables can be compared.

\citet{Mauraya2003c}

\citet{RamMaurya2004}:

\section{Graph search methods}

\subsection{Depth-first search}



%\citet{Maurya2006}
%\citet{Musulin2013}

\section{Ranking algorithms}

%\citet{Farenzena2009}
%\citet{Gao2011}
%\citep{Bryan2006}

\section{Transfer entropy}
%\citet{Schreiber2000}
%\citet{Bauer2005a} (PhD Thesis)
%\citet{Shu2013}


%%A number of data-driven methods for determining causality between control loops have been reported in the past.
%%A relatively new method called transfer entropy has been show to produce robust results even in the absence of observable time delays~\cite{Bauer2007}.
%%Another prominent method uses cross-correlation to estimate time delays in order to infer the sequence of events and therefore causality~\cite{Bauer2008}.
%%Assigning causality based on these methods follows statistical hypothesis testing whereby the apparent degree of correlation is compared to threshold values obtained by studying correlations between random signals of various sampling length and the desired confidence.
%%
%%\citet{Yim2006} reported on the development of a software package combining plant topology with data-driven performance assessment analysis.
%%It is proposed that a similar hybrid approach is followed in order to generate connectivity matrices for the ranking problem using the data-driven methods mentioned above.
%
%%In the Google PageRank algorithm the value of an edge between nodes on a directed graph is considered to be a binary state.
%%In order to refine the sensitivity by which control loops with the greatest sphere of influence can be isolated it is desired to assign some weight instead.
%%These weights can be derived from the causality measured mentioned above.
%%Different weighing strategies will be compared to binary mapping in order to find balance between robust results and engineering effort.
%
%%Tight mass and heat integration on large-scale plants lead to significant control loop interaction.
%%Plants with recycle also commonly suffer from control loop interaction~\citep{Bauer2005}.
%
%%Motivations for data-driven root cause plant-wide disturbance and fault detection include (1) streamlining troubleshooting, reducing man hour requirements, (2) yield information to enhance maintenance efforts during plant shutdown and (3) discover problems not found with traditional "fight today's fire" approaches~\citep{Bauer2005}.
%
%%Fault detection and diagnosis of plant-wide disturbances can be approached either by model-based or data-driven methods.
%%Data-driven methods are generally preferred accurate process models are rarely available - it is estimated that only about 5\% of processes are modeled~\citep{Bauer2005}. % Secondary reference from conference oral presentation
%
%%Model-based techniques compare model parameters against measurements, while data-driven methods are based on historical process data.
%%
%%Measuring the degree of interaction between control loops and ranking them accordingly is considered to be an important yet challenging task~\citep{Rahman2011}.
%%
%%Although a  number of methods have been proposed and tested with varying degrees of success, most require plants to be step tested in order to determine useful interaction measures.
%%Step testing can be costly and in the case of already unstable plants near impossible.
%%Management will rarely authorize step testing to be done, and with good reason.
%%
%%It is suggested that by incorporating interconnectivity information contained in pipeline and instrumentation diagrams (P\&IDs) together with the application of novel interaction algorithms as well as the application of multivariate statistics useful metrics can be derived with none or minimal induced plant disturbances, provided that plant historian data of sufficient resolution (short enough sampling interval) can be accessed.
%
%In addition to finding the root cause of problems, performing economical analysis on identified errors based on known interactions can potentially aid in prioritizing control loop maintenance projects.
%This will particularly be useful for identifying the most important control loop maintenance activities that will require significant process equipment modifications instead of simply altering parameters used by software as these projects are typically associated with higher expenditure.
%
%
%%\section{Literature Study}

% TODO: Study the following first of all and discuss as you go along:

% Bauer2005
% Bauer2008
% Rossi2006
% Gao2011
% Padmanabhan

% Look into the possibility of using Mark Transell's method of finding similar patterns in historical plant data.

% Bauer2005

\citet{Bauer2005} discussed a method to detect the plant-wide propagation path of disturbances on a plant using historical process data.
These methods might be useful for developing a ranking methodology that does not require plant step-testing in order to generate data.

%The emphasis is on plant-wide oscillations caused by disturbances.
%The newly developed method of transfer entropy is introduced to identify the direction of the disturbance, even in the absence of a time delay.

% This is a bit too much word for word, but it is stated so beautifully...
According to \citet{Bauer2005} there is a need for industrial tools that will help company expertise focus on problem areas early and combine data-driven analysis with their knowledge and understanding of the process.
The methods presented were successful in pointing out areas for investigation but site expertise is still required to get to the bottom of them.
There is an urgent requirement for tools that can link process understanding with a data driven analysis which can help company experts focus on getting early solutions to plant problems.

It is the aim of this project to incorporate known physical and logical plant interconnections with multivariate statistical methods to provide metrics that are more clearly defined and rely less on plant knowledge than the previous tools made available.
In a sense the aim is to use the structure of the plant contained in P\&IDs and mimic the thought process of an experienced engineer that is used to produce results that are commonly considered to have relied on intuition gained over many years of plant experience.
Logical as well as physical connections will need to be defined and some of the experienced engineers might be needed to provide information on logical connections that cannot be deducted from P\&IDs.
This should not only help experienced engineers to work more effectively and faster, but will also allow control engineers who are not familiar with all the intricate details of a particular plant to work more independently on a control problem.
Many of the interactions among processes are common and can be classified to generate a set of heuristics that can allow a control engineer to work in a wide variety of environments.

To date there are no data-driven signatures that uniquely identify a plant-wide disturbance as originating from one or another of these categories.
The present state of the art in controller diagnosis is that the process control engineers have to come up with an explanation that is consistent with both the data-driven results and their process understanding.

%Recycling acts as a physical feedback loop.
%A recycle stream can oscillate just as a feedback control loop can oscillate if the gain is too high.
%Process loop gain...

Comparison of causal map from transfer entropy analysis and that generated by logical and physical connections might indicate some undesired control loop interactions.
%Why would we want to go through the trouble of inputting P\&IDs if causal maps from transfer entropy analysis gives a more intricate connection among variables?


\subsection{Transfer entropy}

% Bookmarked Google Books reference...

Causal qualitative models - identify the order of occurrence of events and specify the paths of fault propagation.
Trace root cause of disturbance along propagation path.

Automatic construction of digraphs.

Consistency question:
How does directionality change with faults?

Expected directionality vs. actual directionality.

The combination of plant topology and PDA algortihms can be used to verify that a feasible path between a candidate root cause and other locations in the plant were secondary disturbances have been detected exist ~\cite{Yim2006}.

Signal-based analysis is enhanced by the capture and integration of cause and effect information from a process schematic~\cite{Yim2006}.

Disturbance rejection is indicated when the OP is a member of a disturbed cluster while the PV is not~\cite{Yim2006}.

Manual, ratio, feedforward and cascade configurations can also be detected via rules exploiting the integration of data-drive analysis with plant topologu~\cite{Yim2006}.

The maintenance of accurate drawings is a considerably easier and more routine task than the maintenance of a mathematical model or SDG~\cite{Yim2006}.

``The fuction of CAPEX Plant Analyser is to help engineers form and test hypotheses.''~\cite{Yim2006}

``Standard performance measures, such as control loop performance indices, are not able to deal with plant-wide problems becuase they deal with individual loops one at a time and do not take account of propagation.''~\cite{Bauer2008}

%% Notes of reading on 05/09/13

% Reading from ~\cite{Bauer2005a} (Data-driven methods for process analysis thesis)

Three methods are investigated: cross-correlation, nearest neighbors methods and transfer entropy.
The last two mentioned methods are statistics based.

Root cause analysis is concerned with finding the process measurement that is closets to the originating point of the disturbance.

Different categories of plant wide disturbance root causes include process and constraint problems, controller tuning problems and valve problems. % From reference [21]


% End ~\cite{Bauer2005a}



%% End 05/09/13

The plant-wide analysis is often grouped into two categories: detection and diagnosis.~\cite{Bauer2008} % This is a secondary reference, get original [3].
This work will focus on diagnosis.

A model-free causal analysis has been called for~\cite{Bauer2008} % See original references [6] and [7].

Background:

Holland and Pearl introduced the area known as learning causality from data.

Conditional probabilities - Bayesian networks.
Transfer entropy adds information about time to Bayesian networks because it tests hypotheses concerning the joint and conditional probabilities of past and current values in a time series.





% Nam et al reference for creating digraphs

Diagraph based models usually express the relationship between faults and symptoms and define the propagation paths by incorporating process knowledge of expers.
A drawback is that extracting expert knowledge is very time consuming and the knowledge not always available.

Complete descriptions of chemical plants in terms of differential and algebraic equations are rarely available.

% Get Chiang and Braatz reference

Assumptions:
Process is ergodic.
Ergodic means that the same behaviour averaged over time as averaged over the space of all the system's states (phase space).
The system satisfies the ergodic hypothesis of thermodynamics.

The ergodic hypothesis is often assumed in statistical analysis and states that the time spent by a particle in some region of the phase space of microstates with the same energy is proportional to the volume of this region.
Liouville's theorem ensures that the motion of time average makes sense.

\begin{equation}
  T_{Y\rightarrow X} = \sum_{x_{i+1}}
%\sum_{\bold{x}^k_i}
%\sum_{\bold{y}^l_i} p
%\left(x_{i + 1}, \bold{x}_i^k, \bold{y}^l_i\right)\log{\frac{p \left(x_{i + 1} | \bold{x}_i^k, \bold{y}^l_i \right )}{p \left(x_{i + 1} | \bold{x}_i^k \right )}}
\end{equation}



http://store.elsevier.com/product.jsp?isbn=9780080442976c



% Bauer2005
% Schreiber (2000)




\subsection{Visual inspection}

\subsubsection{Normalized time trends}

Normalized time trends of set points (SP), controlled variables (CV), controller error and controller output (MV) can be useful for a first-pass visual inspection.

Normalized: mean values of time trends removed and standard deviations scaled to unity.

\subsubsection{Spectral analysis}

A plant-wide spectral analysis can detect measurements having similar spectral features.
Measurements whose spectra are similar are likely to be subject to the same disturbance.
Power spectra are invariant with respect to the phase of a signal therefore time delays between tags do not affect the analysis.

Measuring the degree of interaction between control loops and ranking them accordingly is considered to be an important yet challenging task~\citep{Rahman2011}.
The difficulty arises from the sheer number of control loops connected in a multiple-input multiple-output (MIMO) configuration.

Plant engineers are generally overloaded and do not know where the best place is to start the maintenance work~\citep{Rahman2011}.

Various methods for determining control loop interaction measures have been proposed and adopted with varying degrees of success, including~\citep{Rahman2011}: % Also mention Julali article

\begin{itemize}
  \item Standard deviation of error signals
  \item Key performance indicators (KPIs) 
  \item Economic indicators
  \item Oscillation index
  \item Interaction index
\end{itemize}

This study will focus on interaction indexes. % Alternative plural: indices (less American)

The interaction methods can be divided into those based on (1) model identification and (2) data processing.
Due to the lack of accurate models in industry, model based interaction determination methods have found limited use to date.
Data-driven methods have been more successful and methods that do not rely on step-tests but can instead make use of routine operating data or set point change data have been developed~\citep{Rahman2011}. % NB!! Read these articles ASAP.

\subsection{Statistical methods}

%\subsubsection{Auto correlation}
%
%Auto correlation refers to the degree to which data is correlated with itself, in other words how much future values are dependent on historical values. % Confirm this statement!!
%
%An auto correlation value is reported, with 1 indicating the strongest possible correlation and 0 indicating no correlation.
%
%\subsubsection{Cross correlation}
%
%Cross correlation is similar to auto correlation except that the correlation between two different variables is studied.
%
%Covariance...
%
%\subsubsection{Partial correlation}
%
%
%\subsubsection{Local gain}
%
%
%
%\subsubsection{Canonical correlation}
%
%\citet{Rahman2011} reported on a novel use of canonical correlation measures to rank control loops based on interaction.
%This method requires step-test data.
%
%Canonical correlation can be used to quantify relations between multidimensional variables.
%It is assumed that the relationship is linear.
%
%% This section chiefly copied from {Rahman2011}.
%% Ask about suggestions to comply with plagiarism standards. There are only so many technical terms available, especially in math!
%
%The most important difference between canonical correlation analysis (CCA) and ordinary correlation analysis is the fact that they are invariant with respect to affine transformations of the variables.
%
%Affine transformations refers to... % Discuss what this is and why this is important, especially from the interaction analysis.
%% See Weenink 2003 reference

%\subsection{Error criteria}
%
%Interaction measures based on squared error criteria are based on the idea that a change in any parameter of one control loop will increase the integrated square error (ISA) or integrated absolute error (IAE) of any other loop affected by the adjusted loop.
%The set-point is generally used as the parameter that is changed.
%
%\citet{Rahmann2011} suggested the calculation of a loop interaction metric followed by the arrangement of a loop interaction matrix.
%This loop interaction matrix can then be used to calculate a loop importance index (LII).
%
%\citet{Rossi2006} reported on a control loop performance diagnostic method designed for the purpose of determining whether a decentralized PI(D)\textsuperscript{+} controller will be sufficient or if an advanced control structure such as an model predictive controller (MPC) will be needed.
%
%\subsubsection{Integral squared error methods}
%
%
%\subsubsection{Absolute squared error methods}
%
%\citet{Rahman2011} mention that in their experience a step size of 1-2\% of the nominal value is sufficient, depending on noise levels.
%
%It might be necessary to filter noise from data before analysis.
%
%Interaction quantification techniques can be supplemented using cause-effect analysis techniques based on mutual entropy methods that are presented by Bauer and Thornhill (2008, 2005).

%\subsubsection{Error power spectrum}










\subsection{LoopRank}

Farenzena and Trierweiler~\cite{Farenzena2009} were the first to report on the usage of a Google PageRank type algorithm for analysing interactions between control loops, dubbing the method LoopRank.

Wilken~\cite{Wilken2012} studied the application of the Google PageRank type algorithm on the well-known Tennessee Eastman Process.
%It was noted that % insert method here
%was a better interaction metric than
%% insert method here.
An approach was followed whereby a base case set of interaction measures were determined which allowed for future deviations in performance due to maintenance requirements to be pinpointed.
Although this will certainly be of some use in industry, very few industrial plants are running or have run in a manner that can be considered to represent a desirable base case scenario for future comparison.

However, tracking the way control loop interactions change over time will still be useful in monitoring the effect of maintenance activities on the plant, and can possibly indicate which maintenance activities are of the highest priority.

%\subsection{RGA}
%
%The relative gain array introduced by Bristol in 1966 % include reference
%was one of the first interaction measures proposed to provide a metric for the pairing of controlled and manipulated variables.
%
%Originally the RGA was limited to a study of steady-state interactions but its use was later expanded to all frequencies to provide a measure known as the dynamic RGA. % include reference

%\subsection{Open loop responses}
%
%Tung and Edgar's method for determining interaction based on open loop responses. % Find out what it was about.
%
%On many occasions it is important to relate closed-loop responses to open-loop responses.
%A number of methods have been developed to allow for this based on a known controller model.
%% What if no controller model is available or if the model is very complex such as with the case of MPC?

%\subsection{Singular perturbation technique}
%
%% Gagnepain and Seborg (1982)
%% Get reference from ~\cite{Rahman2011}
%
%% Read and discuss all the methods mentioned in ~\cite{Rahmann2011}.
%
%\newpage
%\section{Preliminary literature study}
%
%\subsection{Ranking algorithm}
%
%A modified form of the Google PageRank algorithm \citep{Bryan2006} is used to rank control loops based on their connectivity, interaction and importance scores (weights).
%This concept has been introduced by \citet{Farenzena2009} who dubbed the algorithm LoopRank when applied to the ranking of control loops.
%
%In order to apply the method a directed graph defining the interactions between process variables is needed.
%%For a review of directed graphs, see Narsingh (1974).
%A convenient method for representing a binary directed graph is an adjacency matrix as defined in Eq. \ref{eq:adef}.
%
%\begin{equation} \label{eq:adef}
%A_{ij} = \left( \begin{array}{ll}
%1 & \textrm{If $v_j$ has an edge directed towards $v_i$} \\
%0 &\textrm{Otherwise} \end{array} \right)
%\end{equation}
%
%The importance of a node $x_k$ depends on the importance of the nodes pointing towards $x_k$ as displayed in Eq. \ref{eq:importance}.
%
%\begin{equation} \label{eq:importance}
%x_k = \sum_{j \in \boldsymbol{L}_k} e_j x_j
%\end{equation}
%
%In Eq. \ref{eq:importance} $\boldsymbol{L}_k$ is the set of nodes which have an incident edge to node $x_k$ and the weight $e_j$ ensures that node $x_j$ contributes importance to node $x_k$ in proportion to the extent to which it affects node $x_k$.
%
%If this system is expressed in matrix form the ranking problem reduces to the standard eigenvector problem of Eq. \ref{eq:standard}
%
%\begin{equation} \label{eq:standard}
%\boldsymbol{Ax} = \boldsymbol{x}
%\end{equation}
%
%The $\boldsymbol{A}$ matrix is a normalized adjacency matrix such that each column sums to unity.
%The importance scores of the nodes are the elements of the normalized eigenvector corresponding to the eigenvalue of one in Eq. \ref{eq:standard}.
%
%\subsection{Methods for determining causality}
%
%The ranking algorithm requires a suitable connectivity matrix to be identified.
%Setting up this matrix involves 1) determining the causal connections between process variables and 2) assigning weights and importance scores to the connections between process variables as well as the variables themselves.
%
%The connectivity between process variables can be determined in at least two different ways: 1) defining connections by making use of known physical and logical connectivity information and 2) determining causality using data-driven methods.
%Knowledge-based methods for determining causality are generally limited to generating a binary adjacency matrix and are commonly used in combination with data-driven techniques.
%If knowledge-based methods are combined with data driven methods the main contribution of the knowledge-based methods will be to ensure that the network structure is correct while the data driven methods can be used to calculate appropriate weights for the connections.
%
%Knowledge-based methods seek to determine causal connections by employing information on physical and logical connections coupled with reasoning algorithms.
%\citet{Yim2006} reported on the development of a software package combining plant topology information captured using packages compatible with the Computer Aided Engineering Exchange (CAEX) format and a reasoning engine written in Prolog to generate a signed directed graph (SDG) of the network of tags involved.
%The SDG is then augmented by the use of a plant disturbance analyzer to produce an effective hybrid approach for generating the required connectivity matrix.
%
%In data-driven methods historical data is analyzed using various statistical methods in order to identify probable causal relationships. Methods reported in literature include partial correlation \citep{Farenzena2009}, cross correlation time delay estimation \citep{Bauer2008}, nearest neighbors \citep{Bauer2005} as well as the relatively new concept of transfer entropy first introduced by \citet{Schreiber2000} and applied by \citet{Bauer2007}.
%The significance levels are compared to threshold values obtained by analyzing surrogate random time series data and selecting a required minimum deviation, usually six sigma \citep{Bauer2005}.
%
%The transfer entropy method has been shown to be the most robust compared to the cross-correlation and nearest neighbor methods and is also useful in the absence of noticeable time delays between variables \citep{Bauer2005}.
%\citet{Bauer2005} proposed a modified transfer entropy calculation that allows for estimating the dead time.
%\citet{Shu2013} proposed an additional modification to this method that has been shown to be more accurate and also reported how the obtained time delays can be used to eliminate redundant connections.
%
%\newpage
%\section{Problem statement and research objectives}
%
%Connectivity, historical data and economical attributes should be combined to identify control loops with the greatest influence on profitability and/or stability.
%This will allow more efficient use of control engineer and instrumentation specialist man hours, in turn leading to safer and more economical plant operation.
%
%A good ranking algorithm will identify under-performing control loops and rank them in such an order that the control loop which is affecting the most important variables in the system will be given higher priority than a control loop which may be under-performing to a greater extent but affecting less important variables.
%
%It is proposed that by combining the information contained in plant pipeline and instrumentation diagrams (P\&IDs) as well as employing measures to assign a metric of interaction between control loops, the important and time consuming task of addressing the root cause instead of the symptoms can be significantly aided.
%
%Comparison of causal map from transfer entropy analysis and that generated by logical and physical connections might indicate some undesired control loop interactions.
%
%\newpage
%\section{Expected contributions}
%
%The main contribution of this research project is the integration of various ranking, causal mapping and control loop assessment methods into a tool that can prioritize control loop maintenance on a plant-wide or even site-wide scale.
%
%The main output is a definite, regularly updated list of the most important control loops in need of attention that can be forwarded to base layer control engineers and instrumentation specialists.
%
%The provision of a connectivity and interaction metric that can assist in troubleshooting procedures that previously relied on intuition acquired by experience is expected to be a significant aid in maintaining contingency with respect to plant operations.
%This is of special interest in cases were the previous generation of operators is retiring in rapid succession.
%
%\newpage
%\section{Research strategy}
%
%\subsection{General approach}
%
%In order to prioritize control loops it is necessary to identify control loops with the largest influence which are performing the worst while having the greatest impact on profitability.
%
%The ranking algorithm is a generalized eigenvector problem once a suitable connectivity matrix has been identified. Setting up this matrix involves 1) determining the connectivity between process variables and 2) assigning weights and importance scores to the connections between process variables as well as the variables themselves.
%
%A metric for the influence of a control loop on a plant-wide scale can be determined by modeling the plant as a directed graph where the nodes represent the variables and the edge connectivity is inferred from the physical and logical structure of the plant.
%This directed graph may then be used to prioritize control loop maintenance by ranking the relative importance of the nodes in the system using eigenvector analysis.
%If the actual connectivity is not available, it may be possible to infer it from routine plant data or through direct testing.
%Methods of determining causality between variables on a plant-wide scale have been an active research topic with various papers being published in literature over the past few years.
%
%As the objective is not only to identify control loops that have the greatest influence, but to prioritize loops for maintenance, a performance metric is associated with each control loop.
%As many industrial operations already employ performance monitoring software the plan is to integrate the connectivity with predetermined performance scores.
%In addition, safety and profitability triage metrics should be associated with each control loop.
%Depending on the amount of information available this can be either categorical or numerical.
%
%It is proposed that a modified form of the Google PageRank algorithm be used to rank control loops based on their connectivity, interaction and importance scores (weights). This concept has been introduced by \citet{Farenzena2009} who dubbed the method LoopRank. The application of this method to identify the root cause of induced faults of the simulated Tennessee Eastman problem has been successfully demonstrated~\cite{Wilken2012}.
%
%\citet{Wilken2012} reported a number of anomalies in the calculation of importance scores if the PageRank algorithm when applied in unaltered form to infer importance. These issues were related to an inability to differentiate importance of inputs as well as assigning equal importance to parallel paths. Although workarounds have been proposed, further research into conditions relating to unrealistic results and potential solutions is warranted to increase the robustness of the ranking method.
%
%
%\subsection{Case studies}
%
%The success and usefulness of the integrated holistic approach to control loop maintenance prioritization will be analyzed by simulated scenarios, industrial case studies and interviews with process experts.
%
%Numerous case studies will be carried out in order to evaluate the efficiency and industrial usefulness of the interaction metrics developed.
%A number of custom designed scenarios will likely be used in the initial stages of the project to assess the feasibility of methods and compare results over a wide range of potential problems.
%Well-known academic example problems such as the Shell control problem as well as the Tennessee-Eastman plant will be studied.
%
%Industrial case studies and interviews with process experts will be of significant value in verifying the usefulness and applicability of the developed method.
%Data sourced from industrial operations will likely be denatured to allow for adherence to intellectual property protocols.
%However, it should still be useful to determine the real-life usefulness of the developed methodologies in practical industry, and will allow for the methods to be exposed to the difficulties surrounding industry problems, such as limited historian information or low stored sampling intervals and historian data compression.


%\section{Conclusion}

%\backmatter
%\section{References and preliminary bibliography}

\newpage
\bibliography{library}
%\bibliographystyle{plainnat}
\bibliographystyle{agsm}

%\section*{Personal information}
%
%\begin{tabbing}
%  Telephone number: \= 072 712 6845 \= \kill
%  Full name: \> Simon Jacobus Streicher \\[2ex]
%  Postal address: \> 15 Van Standen Street \\ \> Vaalpark \\ \>1947 \\[2ex] 
%  E-mail address: \> streichersj@gmail.com \\[2ex]
%  Telephone number: \> 072 712 6845 \\[2ex]
%  Academic record: \> B.Eng.(Chem) (NWU) (\textit{cum laude})
%\end{tabbing}

\end{document}


