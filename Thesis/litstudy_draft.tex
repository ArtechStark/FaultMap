\section{Literature study}

% TODO: Study the following first of all and discuss as you go along:

% Bauer2005
% Bauer2008
% Rossi2006
% Gao2011
% Padmanabhan

% Look into the possibility of using Mark Transell's method of finding similar patterns in historical plant data.

% Bauer2005

\citet{Bauer2005} discussed a method to detect the plant-wide propagation path of disturbances on a plant using historical process data.
These methods might be useful for developing a ranking methodology that does not require plant step-testing in order to generate data.

%The emphasis is on plant-wide oscillations caused by disturbances.
%The newly developed method of transfer entropy is introduced to identify the direction of the disturbance, even in the absence of a time delay.

% This is a bit too much word for word, but it is stated so beautifully...
According to \citet{Bauer2005} there is a need for industrial tools that will help company expertise focus on problem areas early and combine data-driven analysis with their knowledge and understanding of the process.
The methods presented were successful in pointing out areas for investigation but site expertise is still required to get to the bottom of them.
There is an urgent requirement for tools that can link process understanding with a data driven analysis which can help company experts focus on getting early solutions to plant problems.

It is the aim of this project to incorporate known physical and logical plant interconnections with multivariate statistical methods to provide metrics that are more clearly defined and rely less on plant knowledge than the previous tools made available.
In a sense the aim is to use the structure of the plant contained in P\&IDs and mimic the thought process of an experienced engineer that is used to produce results that are commonly considered to have relied on intuition gained over many years of plant experience.
Logical as well as physical connections will need to be defined and some of the experienced engineers might be needed to provide information on logical connections that cannot be deducted from P\&IDs.
This should not only help experienced engineers to work more effectively and faster, but will also allow control engineers who are not familiar with all the intricate details of a particular plant to work more independently on a control problem.
Many of the interactions among processes are common and can be classified to generate a set of heuristics that can allow a control engineer to work in a wide variety of environments.

To date there are no data-driven signatures that uniquely identify a plant-wide disturbance as originating from one or another of these categories.
The present state of the art in controller diagnosis is that the process control engineers have to come up with an explanation that is consistent with both the data-driven results and their process understanding.

%Recycling acts as a physical feedback loop.
%A recycle stream can oscillate just as a feedback control loop can oscillate if the gain is too high.
%Process loop gain...

Comparison of causal map from transfer entropy analysis and that generated by logical and physical connections might indicate some undesired control loop interactions.
%Why would we want to go through the trouble of inputting P\&IDs if causal maps from transfer entropy analysis gives a more intricate connection among variables?


\subsection{Transfer entropy}

% Bookmarked Google Books reference...

Causal qualitative models - identify the order of occurrence of events and specify the paths of fault propagation.
Trace root cause of disturbance along propagation path.

Automatic construction of digraphs.

Consistency question:
How does directionality change with faults?

Expected directionality vs. actual directionality.

The combination of plant topology and PDA algortihms can be used to verify that a feasible path between a candidate root cause and other locations in the plant were secondary disturbances have been detected exist ~\cite{Yim2006}.

Signal-based analysis is enhanced by the capture and integration of cause and effect information from a process schematic~\cite{Yim2006}.

Disturbance rejection is indicated when the OP is a member of a disturbed cluster while the PV is not~\cite{Yim2006}.

Manual, ratio, feedforward and cascade configurations can also be detected via rules exploiting the integration of data-drive analysis with plant topologu~\cite{Yim2006}.

The maintenance of accurate drawings is a considerably easier and more routine task than the maintenance of a mathematical model or SDG~\cite{Yim2006}.

``The fuction of CAPEX Plant Analyser is to help engineers form and test hypotheses.''~\cite{Yim2006}

``Standard performance measures, such as control loop performance indices, are not able to deal with plant-wide problems becuase they deal with individual loops one at a time and do not take account of propagation.''~\cite{Bauer2008}

%% Notes of reading on 05/09/13

% Reading from ~\cite{Bauer2005a} (Data-driven methods for process analysis thesis)

Three methods are investigated: cross-correlation, nearest neighbors methods and transfer entropy.
The last two mentioned methods are statistics based.

Root cause analysis is concerned with finding the process measurement that is closets to the originating point of the disturbance.

Different categories of plant wide disturbance root causes include process and constraint problems, controller tuning problems and valve problems. % From reference [21]


% End ~\cite{Bauer2005a}



%% End 05/09/13

The plant-wide analysis is often grouped into two categories: detection and diagnosis.~\cite{Bauer2008} % This is a secondary reference, get original [3].
This work will focus on diagnosis.

A model-free causal analysis has been called for~\cite{Bauer2008} % See original references [6] and [7].

Background:

Holland and Pearl introduced the area known as learning causality from data.

Conditional probabilities - Bayesian networks.
Transfer entropy adds information about time to Bayesian networks because it tests hypotheses concerning the joint and conditional probabilities of past and current values in a time series.





% Nam et al reference for creating digraphs

Diagraph based models usually express the relationship between faults and symptoms and define the propagation paths by incorporating process knowledge of expers.
A drawback is that extracting expert knowledge is very time consuming and the knowledge not always available.

Complete descriptions of chemical plants in terms of differential and algebraic equations are rarely available.

% Get Chiang and Braatz reference

Assumptions:
Process is ergodic.
Ergodic means that the same behaviour averaged over time as averaged over the space of all the system's states (phase space).
The system satisfies the ergodic hypothesis of thermodynamics.

The ergodic hypothesis is often assumed in statistical analysis and states that the time spent by a particle in some region of the phase space of microstates with the same energy is proportional to the volume of this region.
Liouville's theorem ensures that the motion of time average makes sense.

\begin{equation}
  T_{Y\rightarrow X} = \sum_{x_{i+1}}
%\sum_{\bold{x}^k_i}
%\sum_{\bold{y}^l_i} p
%\left(x_{i + 1}, \bold{x}_i^k, \bold{y}^l_i\right)\log{\frac{p \left(x_{i + 1} | \bold{x}_i^k, \bold{y}^l_i \right )}{p \left(x_{i + 1} | \bold{x}_i^k \right )}}
\end{equation}



http://store.elsevier.com/product.jsp?isbn=9780080442976c



% Bauer2005
% Schreiber (2000)




\subsection{Visual inspection}

\subsubsection{Normalized time trends}

Normalized time trends of set points (SP), controlled variables (CV), controller error and controller output (MV) can be useful for a first-pass visual inspection.

Normalized: mean values of time trends removed and standard deviations scaled to unity.

\subsubsection{Spectral analysis}

A plant-wide spectral analysis can detect measurements having similar spectral features.
Measurements whose spectra are similar are likely to be subject to the same disturbance.
Power spectra are invariant with respect to the phase of a signal therefore time delays between tags do not affect the analysis.

Measuring the degree of interaction between control loops and ranking them accordingly is considered to be an important yet challenging task~\citep{Rahman2011}.
The difficulty arises from the sheer number of control loops connected in a multiple-input multiple-output (MIMO) configuration.

Plant engineers are generally overloaded and do not know where the best place is to start the maintenance work~\citep{Rahman2011}.

Various methods for determining control loop interaction measures have been proposed and adopted with varying degrees of success, including~\citep{Rahman2011}: % Also mention Julali article

\begin{itemize}
  \item Standard deviation of error signals
  \item Key performance indicators (KPIs) 
  \item Economic indicators
  \item Oscillation index
  \item Interaction index
\end{itemize}

This study will focus on interaction indexes. % Alternative plural: indices (less American)

The interaction methods can be divided into those based on (1) model identification and (2) data processing.
Due to the lack of accurate models in industry, model based interaction determination methods have found limited use to date.
Data-driven methods have been more successful and methods that do not rely on step-tests but can instead make use of routine operating data or set point change data have been developed~\citep{Rahman2011}. % NB!! Read these articles ASAP.

\subsection{Statistical methods}

%\subsubsection{Auto correlation}
%
%Auto correlation refers to the degree to which data is correlated with itself, in other words how much future values are dependent on historical values. % Confirm this statement!!
%
%An auto correlation value is reported, with 1 indicating the strongest possible correlation and 0 indicating no correlation.
%
%\subsubsection{Cross correlation}
%
%Cross correlation is similar to auto correlation except that the correlation between two different variables is studied.
%
%Covariance...
%
%\subsubsection{Partial correlation}
%
%
%\subsubsection{Local gain}
%
%
%
%\subsubsection{Canonical correlation}
%
%\citet{Rahman2011} reported on a novel use of canonical correlation measures to rank control loops based on interaction.
%This method requires step-test data.
%
%Canonical correlation can be used to quantify relations between multidimensional variables.
%It is assumed that the relationship is linear.
%
%% This section chiefly copied from {Rahman2011}.
%% Ask about suggestions to comply with plagiarism standards. There are only so many technical terms available, especially in math!
%
%The most important difference between canonical correlation analysis (CCA) and ordinary correlation analysis is the fact that they are invariant with respect to affine transformations of the variables.
%
%Affine transformations refers to... % Discuss what this is and why this is important, especially from the interaction analysis.
%% See Weenink 2003 reference

%\subsection{Error criteria}
%
%Interaction measures based on squared error criteria are based on the idea that a change in any parameter of one control loop will increase the integrated square error (ISA) or integrated absolute error (IAE) of any other loop affected by the adjusted loop.
%The set-point is generally used as the parameter that is changed.
%
%\citet{Rahmann2011} suggested the calculation of a loop interaction metric followed by the arrangement of a loop interaction matrix.
%This loop interaction matrix can then be used to calculate a loop importance index (LII).
%
%\citet{Rossi2006} reported on a control loop performance diagnostic method designed for the purpose of determining whether a decentralized PI(D)\textsuperscript{+} controller will be sufficient or if an advanced control structure such as an model predictive controller (MPC) will be needed.
%
%\subsubsection{Integral squared error methods}
%
%
%\subsubsection{Absolute squared error methods}
%
%\citet{Rahman2011} mention that in their experience a step size of 1-2\% of the nominal value is sufficient, depending on noise levels.
%
%It might be necessary to filter noise from data before analysis.
%
%Interaction quantification techniques can be supplemented using cause-effect analysis techniques based on mutual entropy methods that are presented by Bauer and Thornhill (2008, 2005).

%\subsubsection{Error power spectrum}










\subsection{LoopRank}

Farenzena and Trierweiler~\cite{Farenzena2009} were the first to report on the usage of a Google PageRank type algorithm for analysing interactions between control loops, dubbing the method LoopRank.

Wilken~\cite{Wilken2012} studied the application of the Google PageRank type algorithm on the well-known Tennessee Eastman Process.
%It was noted that % insert method here
%was a better interaction metric than
%% insert method here.
An approach was followed whereby a base case set of interaction measures were determined which allowed for future deviations in performance due to maintenance requirements to be pinpointed.
Although this will certainly be of some use in industry, very few industrial plants are running or have run in a manner that can be considered to represent a desirable base case scenario for future comparison.

However, tracking the way control loop interactions change over time will still be useful in monitoring the effect of maintenance activities on the plant, and can possibly indicate which maintenance activities are of the highest priority.

%\subsection{RGA}
%
%The relative gain array introduced by Bristol in 1966 % include reference
%was one of the first interaction measures proposed to provide a metric for the pairing of controlled and manipulated variables.
%
%Originally the RGA was limited to a study of steady-state interactions but its use was later expanded to all frequencies to provide a measure known as the dynamic RGA. % include reference

%\subsection{Open loop responses}
%
%Tung and Edgar's method for determining interaction based on open loop responses. % Find out what it was about.
%
%On many occasions it is important to relate closed-loop responses to open-loop responses.
%A number of methods have been developed to allow for this based on a known controller model.
%% What if no controller model is available or if the model is very complex such as with the case of MPC?

%\subsection{Singular perturbation technique}
%
%% Gagnepain and Seborg (1982)
%% Get reference from ~\cite{Rahman2011}
%
%% Read and discuss all the methods mentioned in ~\cite{Rahmann2011}.