\documentclass{article}
%\usepackage{syntonly}
\usepackage{fancyhdr}
\usepackage{amsmath}
%\syntaxonly
\usepackage{titling}
%\usepackage{apalike}
\usepackage[authoryear,round]{natbib}
\newcommand{\subtitle}[1]{%
  \posttitle{%
    \par\end{center}
    \begin{center}\Large#1\end{center}
    \vskip0.5em}%
  }
\setlength{\parindent}{0pt}
\setlength{\parskip}{1ex plus 0.5ex minus 0.2ex}
\fancyhf{}
\renewcommand{\headrulewidth}{0pt}
\fancyfoot[LE,RO]{\thepage}

\begin{document}
\pagenumbering{roman}


\title{Eigenvector analysis for the ranking of control loop importance}
\subtitle{One page summary}
\date{January 27, 2014}
\author{Simon Jacobus Streicher}
\maketitle
\thispagestyle{empty}

%\newpage
%\tableofcontents
%\thispagestyle{empty}

%\frontmatter

%\mainmatter


\newpage
\pagenumbering{arabic}
\pagestyle{fancy}

% This is intended to serve as a reminder of the core tasks required in the research project and to limit the amount of time spent investigating other ideas.
% Alterations to this document should be made only after careful consideration.
% Time spent can only be justified as being related to the thesis if it relates to the main ideas presented in this summary.

The main focus point of the thesis is the \textit{prioritization} of base layer control loops from a \textit{maintenance} perspective.
The fields of fault detection, fault root cause analysis and process supervision is closely related to this project.

The goal is to \textit{identify}, \textit{develop}, \textit{implement} and \textit{test} a method that is:

\begin{itemize}
\item Efficient

The method must be correct about which loops are in the greatest need of maintenance based on the economics of the operation.
\item Low maintenance

Both the initial implementation as well as the ongoing application of the method should be low in cost in terms of quantity and skill level of man hours required.
\end{itemize}

In order to be efficient on a plant-wide scale the method will require quality information on:
\begin{itemize}
\item Individual loop performance

Not a main focus point of this thesis as there are many tools available that can provide detailed performance indicators on individual loops.
\item Distributed effect of individual loops in the network

The focus point of this thesis.
To be approached by applying node ranking algorithms to signed directed graphs.
\item Value of the affected streams

Not a main focus point of this thesis but there is certainly room for improvement in this area in general.
% Make this a footnote
%Future work may consider incorporating the business area method as initially developed by S.J. Streicher (snr) which proved to be very useful on a business wide level over the three year period it was used.
\end{itemize}

The main identification and development work will therefore be related to:

\begin{itemize}
\item Finding weighted signed directed graphs (SDGs) of operations
\item Applying efficient ranking algorithms to the SDGs
\end{itemize}

Finding weighted SDGs can be accomplished via:
\begin{itemize}
\item Knowledge based methods

%Network structure usually accurate but mostly limited to binary edge values.
%
%Examples:
%P\&IDs and CAEX compatible drawings combined with reasoning expert engines (Prolog)
%Expert knowledge capture

\item Data driven methods

%Edges can be given a weight based on the strength of the connection.
%
%Cross correlation and transfer entropy are two widely applied methods.
%Transfer entropy has been shown to outperform other methods in various aspects and will be the main method investigated.

\item Hybrid methods

%Hybrid methods usually employ knowledge based methods in order to get an accurate network structure and data driven methods to determine suitable weights for the edges.
\end{itemize}

The development of efficient SDG ranking algorithms have been driven by the information technology industry.
Algorithms will be adapted to the unique features of chemical plants.

The ranking algorithm should:
\begin{itemize}
\item Have an efficient implementation

% Initial attempt to quantify requirements
%Networks of several thousand nodes should not require a computational time of more than several hours on a server with about 10 racks.
%These algorithms do exist and they also happen to include some of the more useful ones.
%See the paper by \citet{Gao2011}

\item Incorporate meta-data about the nodes and edges

%This is known as the semi-supervised learning framework \citep{Gao2011}.
%
%Data about the performance of individual control loops as well as the economical value of streams need to be considered in the ranking algorithm.
\end{itemize}

The implementation of the method will be done in the Python programming language and associated libraries.

The method will be tested using both benchmark plant problems as well as real industrial case studies.

\end{document}


